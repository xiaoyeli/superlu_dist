\begin{figure}[tbp]
\begin{itemize}
\item \begin{verbatim}
A = { Stype = SLU_NC; Dtype = SLU_D; Mtype = SLU_GE; nrow = 5; ncol = 5;
      *Store = { nnz = 12;
                 nzval = [ 19.00, 12.00, 12.00, 21.00, 12.00, 12.00, 21.00,
                           16.00, 21.00, 5.00, 21.00, 18.00 ];
                 rowind = [ 0, 1, 4, 1, 2, 4, 0, 2, 0, 3, 3, 4 ];
                 colptr = [ 0, 3, 6, 8, 10, 12 ];
               }
    }
      \end{verbatim}
\item \begin{verbatim}
U = { Stype = SLU_NC; Dtype = SLU_D; Mtype = SLU_TRU; nrow = 5; ncol = 5;
      *Store = { nnz = 11;
                 nzval = [ 21.00, -13.26, 7.58, 21.00 ];
                 rowind = [ 0, 1, 2, 0 ];
                 colptr = [ 0, 0, 0, 1, 4, 4 ];
               }
    }
      \end{verbatim}
\item \begin{verbatim}
L = { Stype = SLU_SC; Dtype = SLU_D; Mtype = SLU_TRLU; nrow = 5; ncol = 5;
      *Store = { nnz = 11;
                 nsuper = 2;
                 nzval = [ 19.00, 0.63, 0.63, 21.00, 0.57, 0.57, -13.26,
                           23.58, -0.24, 5.00, -0.77, 21.00, 34.20 ];
                 nzval_colptr = [ 0 3, 6, 9, 11, 13 ];
                 rowind = [ 0, 1, 4, 1, 2, 4, 3, 4 ];
                 rowind_colptr = [ 0, 3, 6, 6, 8, 8 ];
                 col_to_sup = [ 0, 1, 1, 2, 2 ];
                 sup_to_col = [ 0, 1, 3, 5 ];
               }
    }
      \end{verbatim}
\end{itemize}
\caption{The data structures for a $5\times 5$ matrix and its $LU$ factors, 
        as represented in the {\tt SuperMatrix} data structure.
	Zero-based indexing is used.}
  \label{fig:matrixeg}
\end{figure}
